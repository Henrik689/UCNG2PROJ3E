A line following robot is essentially a robot designed for the consumer to follow a line or path that is not predetermined. This line or path may be as simple as a strip of tape or a black line and, if  developed further, can follow e.g magnetic markers, embedded lines or laser guided markers. In order to detect the various lines or paths, miscellaneous sensors or sensing methods can be used.\newline These methods may range from simple low cost sensors to advanced and more expensive vision systems, for example cameras. In the industry the many different types of robots are already implemented in semi to fully automatic systems.\\

The project was handed to the group April 12th and will be handed in at UCN Sofiendalsvej, June 7th at 12.00.\newline
The objective of this project is to design and implement an automotive robot capable of autonomous maneuvering, specifically a line-following robot employing light detecting sensors.  \\
The challenges at hand are to design a system for the board, to utilize the ADC capabilities of the chip and to implement a PID controller; furthermore, test the products performance on a test track to optimize the control algorithms by adjusting values and to implement a hardware solution featuring light detecting sensors.