Beskriv test section

\section{Unit Testing}

\subsection{Infrared TBD: Find navn på sensor}


\subsubsection{Equipment}
\begin{enumerate}
    \item[•]Hameg HM8040-2 Triple Power Supply
    \item[•]Agilent MSO-X 3024A Oscilloscope    
\end{enumerate}

\subsubsection{Setup}
Power to the sensor is supplied by the power supply and the output is read on the oscilloscope.

\subsubsection{Results}
\textbf{30mm:}  2.29V\\
\textbf{50mm:}  3.07V\\
\textbf{100mm:} 2.34V\\
\textbf{200mm:} 1.31V\\
\textbf{500mm:} 515mV\\

As is evident in the results above, this sensor does not work at low distances, which is also described in the datasheet.

\subsection{Ultralyd TBD: Find navn på sensor}


\subsubsection{Equipment}
\begin{enumerate}
    \item[•]Arduino UNO
    \item[•]Agilent MSO-X 3024A Oscilloscope    
\end{enumerate}

\subsubsection{Setup}
A small program for the Arduino has been written which allows the MCU to trigger the sensor and waits a pre-specified amount of time before triggering again, to allow for the return of the ultrasound wave. 

\subsubsection{Results}
\textbf{30mm:}  152 us\\
\textbf{50mm:}  285 us\\
\textbf{100mm:} 600 us\\
\textbf{200mm:} 1.23 ms\\
\textbf{500mm:} 2.32 ms\\

\subsection{QRE1113}

\subsubsection{Equipment}
\begin{enumerate}
    \item[•]Hameg HM8040-2 Triple Power Supply
\end{enumerate}

\subsubsection{Setup}
Power to the motors is supplied by the power supply, which also measures the current.

\subsubsection{Results}
Measured current on both engines is between 70 and 80 mA when running freely.

\subsection{DC Motors}

\subsubsection{Equipment}
\begin{enumerate}
	\item[•]Hameg HM8040-2 Triple Power Supply
	\item[•]Fluke 45 Multimeter
\end{enumerate}

\subsubsection{Setup}

\subsubsection{Results}

\subsection{H-Bridge}

\subsubsection{Equipment}
\begin{enumerate}
	\item[•]Elcanic Power Supply
	\item[•]Fluke 45 Multimeter
\end{enumerate}

\subsubsection{Setup}
Power to the H-bridge is supplied by the power supply. Current and voltage will be measured by the multimeter.
All of the Mosfet transistors were removed before doing the test. It was done to avoid powering the engines.


\subsubsection{Results}
There was a suspicion concerning the H-bridge, it had been measured that there was an error or a damaged component. The affected suspected part was the H-bridge for the left motor.\\

All of the transistors were tested on the part of the H-bridge in question.
Since there was a 0 in value on the base, it is suspected that a resistor which sat on the board just before the transistor. This and the other resistor values were measured and found to be identical, based on the measures it could be concluded that it was most likely a failed transistor.\\

It was concluded that the damaged transistor was the: Q14 \cite{Q14}  
(TBD link til schematics).\\

The results were as following:\

Normal NPN transistors: base 0.854V, collector 0.073V\

Faulty NPN transistor. base 0V, collector 0V.\\

\subsection{PWM}

\subsubsection{Equipment}

\subsubsection{Setup}

\subsubsection{Results}

\subsection{ADC}

\subsubsection{Equipment}
 
\subsubsection{Setup}
 
\subsubsection{Results}
 
\section{Integration Testing}

\subsection{PWM motor control}
 
\subsubsection{Equipment}
 
\subsubsection{Setup}
 
\subsubsection{Results}
 
\subsection{Robot to Interface communication}

\subsubsection{Equipment}

\subsubsection{Setup}
 
\subsubsection{Results}
 
\section{System Testing}
 
\subsubsection{Equipment}

\subsubsection{Setup}

\subsubsection{Results}
 
\section{Acceptance Testing}
 
\subsubsection{Equipment}
 
\subsubsection{Setup}
 
\subsubsection{Results}
 




