\subsection{CPLD}
The purpose of this test is to verify that CPLD performs the correct logic task to control the dual h-bridge solutions.
\subsubsection{Equipment}
\begin{enumerate}
	\item[•]Agilent MSO-X 3024A Oscilloscope
	\item[•]Agilent 54620-61601 Logic Analyzer Probe Cable
	\item[•]ChipKit Uno32 Microcontroller
\end{enumerate}

\subsubsection{Setup}
For the test logic probes was used to analyze the logic signals comming from the Uno32 into the CPLD aswell as the outputs from the CPLD going to the h-bridges.

\subsubsection{Results}
To compress the results the output will be in the following format:
\begin{table}[h]
\centering
\scalebox{0.9}{
\begin{tabular}{|l|l|l|l|l|l|l|l|l|l|l|l|}
\hline
\multicolumn{4}{|l|}{\textbf{Logic Input}} & \multicolumn{4}{l|}{\textbf{Expected output}} & \multicolumn{4}{l|}{\textbf{Recieved output}} \\ \hline
En          & PWM       & A       & B      & TL           & BL         & TR      & BR      & TL           & BL         & TR      & BR      \\ \hline
Bit 3       & 2         & 1       & 0      & Bit 3        & 2          & 1       & 0       & Bit 3        & 2          & 1       & 0       \\ \hline
\end{tabular}}
\caption{Results example}
\label{resultsexample}
\end{table}

The results are from testing according to the truth table used to create the logic circuit programmed into the CPLD.
\begin{table}[]
\centering
\scalebox{0.9}{
\begin{tabular}{|l|l|l|l|}
\hline
\textbf{Logic Input} & \textbf{Expected output} & \textbf{Recieved output} & \textbf{Pass} \\ \hline
0000                 & 0101                     & 0101                     & Yes           \\ \hline
0001                 & 0101                     & 0101                     & Yes           \\ \hline
0010                 & 0101                     & 0101                     & Yes           \\ \hline
0011                 & 0101                     & 0101                     & Yes           \\ \hline
0100                 & 0101                     & 0101                     & Yes           \\ \hline
0101                 & 0101                     & 0101                     & Yes           \\ \hline
0110                 & 0101                     & 0101                     & Yes           \\ \hline
0111                 & 0101                     & 0101                     & Yes           \\ \hline
1000                 & 0101                     & 0101                     & Yes           \\ \hline
1001                 & 0101                     & 0101                     & Yes           \\ \hline
1010                 & 0101                     & 0101                     & Yes           \\ \hline
1011                 & 0101                     & 0101                     & Yes           \\ \hline
1100                 & 0000                     & 0000                     & Yes           \\ \hline
1101                 & 0011                     & 0011                     & Yes           \\ \hline
1110                 & 1100                     & 1100                     & Yes           \\ \hline
1111                 & 1111                     & 1111                     & Yes           \\ \hline
\end{tabular}}
\caption{CPLD logic results}
\label{cpldresult}
\end{table}