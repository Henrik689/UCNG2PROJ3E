\section{C code}
main.c:

\begin{lstlisting}
//**** GLOBAL INCLUDES ****//
#include <xc.h>
#include <stdio.h>
#include <stdlib.h>
#include <stdint.h>
#define _SUPPRESS_PLIB_WARNING 1   
#define _DISABLE_OPENADC10_CONFIGPORT_WARNING 1  
#include <plib.h>
#include <math.h>

//**** PROJECT INLCUDES ****//
#include "setup.h"
#include "Functions.h"
#include "vehicle.h"
#include "UART.h"
#include "Delay.h"
#include "ADC.h"

//**** VARIABLE DEFINITIONS ****//
long tach1 = 0, tach2 = 0;
int goalX = 50;
int goalY = 200;
int printTimer = 0;
double angle = 0;
int currentX;
int currentY;
int currentTach = 0;
long Ultrasound[3];
int PWM[] = {25,25};
char str[10+1];
double rollingData[3][5] = {{500,500,500,500,500},
                            {500,500,500,500,500},
                            {500,500,500,500,500}  };
int rollingAverage[3];

//**** INTERRUPT SERVICE ROUTINE ****//
void __ISR(_EXTERNAL_1_VECTOR, IPL4SOFT) INT1Interrupt(){
    asm volatile("di");
    if(PORTDbits.RD10 == 0 && PORTDbits.RD5 == 1)//Right wheel backwards
        tach1--;
    else if(PORTDbits.RD10 == 1 && PORTDbits.RD5 == 0)//Right wheel forwards
        tach1++;
    IFS0bits.INT1IF = 0;
    asm volatile("ei");
}

int main(int argc, char** argv) {
    int i = 0;
    int j = 0;
    
    //**** INITIALIZATIONS ****//
    UART1Init(115200, 1);
    sensorIOInit();
    startEngines();
    initInterrupt();
    initTimer();
    initPWM();
    
    UART1WriteLn("Starting");
    
    adjustDuty(1,PWM[0]);
    adjustDuty(2,PWM[1]);
    initialTurn(goalX,goalY);
    forwards();
    
    for (;;){
        rollingAverage();
        controller(rollingAverage[0],rollingAverage[1],rollingAverage[2]);
    }
}
\end{lstlisting}
\newpage

functions.c
\begin{lstlisting}
#include <xc.h>
#include "Delay.h"
#include <stdint.h>
#include <math.h>

#define Trigger1S TRISGbits.TRISG6
#define Echo1S TRISFbits.TRISF6
#define Trigger2S TRISEbits.TRISE7
#define Echo2S TRISEbits.TRISE2
#define Trigger3S TRISEbits.TRISE3
#define Echo3S TRISDbits.TRISD0

#define Trigger1 PORTGbits.RG6
#define Echo1 PORTFbits.RF6
#define Trigger2 PORTEbits.RE7
#define Echo2 PORTEbits.RE2
#define Trigger3 PORTEbits.RE3
#define Echo3 PORTDbits.RD0


double tachPerCm = 16.667;
int wheelbase = 9;
extern long tach1;
extern long tach2;
extern double angle;
extern int currentX;
extern int currentY;
extern int currentTach;
extern int rollingData;
extern int rollingAverage;

void initInterrupt(){
    INTEnableSystemMultiVectoredInt();
    IEC0bits.INT1IE = 0;
    INTCONbits.INT1EP = 0;
    IPC1bits.INT1IP = 4;
    IEC0bits.INT1IE = 1;
}

void initTimer(){
    T4CON = 0x0;              // Stop any 16/32-bit Timer4 operation 
    T5CON = 0x0;              // Stop any 16-bit Timer5 operation 
    T4CONSET = 0x0038;        // Enable 32-bit mode, prescaler 1:8, internal peripheral clock source
    TMR4 = 0x0;               // Clear contents of the TMR4 and TMR5
    PR4 = 0xFFFFFFFF;          // Load PR4 and PR5 registers with 32-bit value
    T4CONSET = 0x8000;
}

void dec_to_str(char* str, long val, size_t digits) {
    size_t i = 1u;
    for (; i <= digits; i++) {
        str[digits - i] = (char) ((val % 10u) + '0');
        val /= 10u;
    }
    str[i - 1u] = '\0'; // assuming you want null terminated strings?
}

long map(long x, long in_min, long in_max, long out_min, long out_max)
{
  return (x - in_min) * (out_max - out_min) / (in_max - in_min) + out_min;
}

void initPWM() {
    int sysClk = 80000000;
    int pwmFreq = 1000;
    int prescaleV = 1;
    int dutyCycle = 0;
    
    PMCONbits.ON = 0;
    PMAEN = 0;
    
    OC4CON = 0x0000;
    OC4R = 0x00638000;
    OC4RS = 0x00638000;
    OC4CON = 0x0006;
    OC4RS = (PR4 + 1)*((float) dutyCycle / 100);
    OC4CONSET = 0x8020; //Enable peripheral, bit 5: 0=16 bit compare mode 1=32 bit
    
    OC5CON = 0x0000;
    OC5R = 0x00638000;
    OC5RS = 0x00638000;
    OC5CON = 0x0006;
    OC5RS = (PR2 + 1)*((float) dutyCycle / 100);

    T2CONSET = 0x0008;
    T2CONSET = 0x8000;
    
    OC5CONSET = 0x8020;
    
    PR4 = (sysClk / (pwmFreq * 2) * prescaleV) - 1;
    PR2 = (sysClk / (pwmFreq * 2) * prescaleV) - 1;
}

void sensorIOInit(){
    Trigger1S = 0;
    Echo1S = 1;
    Trigger2S = 0;
    Echo2S = 1;
    Trigger3S = 0;
    Echo3S = 1;
}

void adjustDuty(int channel, int duty) {
    switch (channel) {
        case 1:
            OC5RS = (PR2 + 1)*((float) duty / 100);
            break;
        case 2:
            OC4RS = (PR4 + 1)*((float) duty / 100);
            break;
    }
}

long micros(){
    long result = ((TMR5<<16)+(TMR4<<0))/5; //Timer5: last 16 bits in timer, bitshift to get a proper result
    return result;
}

long millis(){
    long result = ((TMR5<<16)+(TMR4<<0))/5;
    return result/1000;
}

long readUltrasonic(int channel){  
    long timerFinish = 0;
    long timerOld = 0;
    int timeout = 3000; 
    switch(channel){
        case 1:
            Trigger1 = 1;
            DelayUs(10);
            Trigger1 = 0;
            while(Echo1 == 0){}
            timerOld = micros();
            while(Echo1 == 1){
                timerFinish = micros();
                if(timerFinish - timerOld > timeout || timerFinish - timerOld < 0)
                    return 500;
            }
            break;
        case 2:
            Trigger2 = 1;
            DelayUs(10);
            Trigger2 = 0;
            while(Echo2 == 0){}
            timerOld = micros();
            while(Echo2 == 1){
                timerFinish = micros();
                if(timerFinish - timerOld > timeout || timerFinish - timerOld < 0)
                    return 500;
            }
            break;
        case 3:
            Trigger3 = 1;
            DelayUs(10);
            Trigger3 = 0;
            while(Echo3 == 0){}
            timerOld = micros();
            while(Echo3 == 1){
                timerFinish = micros();
                if(timerFinish - timerOld > timeout || timerFinish - timerOld < 0)
                    return 500;
            }
            break;
        default:
            UART1Write("Default state");
            break;
    }
    timerFinish = micros();
    long result = ((timerFinish-timerOld)*0.34)/2;
    return result;  
}

double angleToTach(int angle)
{
    return angle*3.14159/180*wheelbase/2*tachPerCm;
}

void initialTurn(int x, int y){
    tach1 = 0;
    char str[10+1];
    double angleRadians = atan(y/x);
    int angleD = angleRadians*180/3.14159; //Convert to degrees
    double tachToTurn = angleToTach(angleD);
    turnLeft();
    while(tach1 < tachToTurn){}
    brake();
    angle = angleRadians;
}

void presetTurnLeft()
{
    int tachTarget = tach1+angleToTach(30); //tachCount right //Calculate how far the wheel must turn to get 30 degrees using the arc length of a circular sector
    DelayMs(50);
    turnLeft();
    while(tachTarget > tach1){}
    forwards();
    DelayMs(50);
}

void presetTurnRight()
{
    int tachTarget = tach1-angleToTach(30); //tachCount right //Calculate how far the wheel must turn to get 30 degrees using the arc length of a circular sector
    while(tachTarget > tach1)
        turnRight();
}

void calculatePosition()
{
    int distanceRun = tach1-currentTach; //How far has it run since last calculation
    currentX = currentX+cos(angle)*distanceRun; //Calculate position on coordinate
    currentY = currentY+sin(angle)*distanceRun;
    currentTach = tach1; //Reset 0-position
}

void rollingAverage(){
    int i;
    int j;
    for(i=0;i<3;i++){ //Run through once per sensor
        double tempData = readUltrasonic(i+1); //Save the data on current sensor
        if(tempData < 10000 && tempData > 0){ //To avoid huge and negative numbers
            rollingData[i][4] = tempData; //Save data on the last spot in the array
            rollingAverage[i]=(rollingData[i][0]+rollingData[i][1]+rollingData[i][2]+rollingData[i][3]+rollingData[i][4])/5; //Calculate average from last 5 readings
        }
        for(j=0;j<4;j++){
            rollingData[i][j]=rollingData[i][j+1]; //Move all data one spot left in the arrays
        }
    }
}

void controller(int midSensor, int rightSensor, int leftSensor){
    int midDistance = 100;
    int sideDistance = 50;
    if(midSensor < midDistance)
    {
        brake();
        calculatePosition();
        backwards();
        int tachTarget = tach1-300;
        while(tach1 > tachTarget){}
        brake();
        presetTurnLeft();
        brake();
        forwards();
    }
    else if(leftSensor < sideDistance)
    {
        int tachTarget = tach1-20;
        turnRight();
        while(tachTarget > tach1){}
        brake();

    }
    else if(rightSensor < sideDistance)
    {
        int tachTarget = tach1+20;
        turnLeft();
        while(tachTarget < tach1){}
        brake();
    }
    else
        forwards();
}
\end{lstlisting}

vehicle.c
\begin{lstlisting}
#include <xc.h>
#include "Delay.h"
#include <stdint.h>

//HB1A
#define HB1ASetup TRISDbits.TRISD10
#define HB1A PORTDbits.RD10
//HB1B
#define HB1BSetup TRISDbits.TRISD5
#define HB1B PORTDbits.RD5
//H1PWM
#define HB1PWMSetup TRISDbits.TRISD3
#define HB1PWM PORTDbits.RD3
//HB1DIR
#define HB1EnableSetup TRISDbits.TRISD11
#define HB1Enable PORTDbits.RD11

//HB2A
#define HB2ASetup TRISDbits.TRISD6
#define HB2A PORTDbits.RD6
//HB2B
#define HB2BSetup TRISGbits.TRISG8
#define HB2B PORTGbits.RG8
//HB2PWM
#define HB2PWMSetup TRISDbits.TRISD4
#define HB2PWM PORTDbits.RD4
//HB2DIR
#define HB2EnableSetup TRISDbits.TRISD7
#define HB2Enable PORTDbits.RD7

void startEngines()
{
    HB1ASetup = 0;
    HB1BSetup = 0;
    HB1EnableSetup = 0;
    
    HB2ASetup = 0;
    HB2BSetup = 0;
    HB2EnableSetup = 0;
    
    HB1Enable= 1;
    HB1A = 0;
    HB1B = 0;
    
    HB2Enable= 1;
    HB2A = 0;
    HB2B = 0;
}
void forwards()
{
    HB1Enable = 0;
    HB2Enable = 0;
    
    HB1A = 1;
    HB1B = 0;
    HB2A = 1;
    HB2B = 0;
    
    HB1Enable = 1;
    HB2Enable = 1;
}
void backwards()
{
    HB1Enable = 0;
    HB2Enable = 0;
    
    HB1A = 0;
    HB1B = 1;
    HB2A = 0;
    HB2B = 1;
    
    HB1Enable = 1;
    HB2Enable = 1;
}
void brake()
{
    HB1Enable = 0;
    HB2Enable = 0;
    
    HB1A = 1;
    HB1B = 1;
    HB2A = 1;
    HB2B = 1;
    
    HB1Enable = 1;
    HB2Enable = 1;
}
void turnLeft()
{
    HB1Enable = 0;
    HB2Enable = 0;

    HB1A = 1;
    HB1B = 0;
    HB2A = 0;
    HB2B = 1;
    
    HB1Enable = 1;
    HB2Enable = 1;
}
void turnRight()
{
    HB1Enable = 0;
    HB2Enable = 0;
    
    HB1A = 0;
    HB1B = 1;
    HB2A = 1;
    HB2B = 0;
    
    HB1Enable = 1;
    HB2Enable = 1;
}
\end{lstlisting}