\section{Description of the hardware structure and functionality}

In this section the different components of the hardware will be listed, described and explained.

\section{Hardware diagram}
Beskrivelse af hardware diagram


\subsection{Object avoidance sensor choices}
SR04 Ultrasound
GP2Y0A02YK0F (long) \\	
GP2Y0A41SK0F (short) \\
\\
\subsection{Line following sensor choice} 

The QRE113 sensor

Due to past experiences, the QRE113 sensor has been chosen to be utilized on the robot to enable its line-following properties. It works by emitting infarred light onto a surface, and then taking a reading based on the amount of light that gets reflected. A light surface will reflect more light back than a dark one. The sensor then regulates it output voltage from 1 to 100%. Based on this output voltage, it is possible to use an ADC to convert these signals into digital signals, which can be monitored more conveniently. Functionally, the robot is left with a way of knowing which surface the sensors are above - and in the case of a track with a black line to follow, this allows it to detect where the line it needs to follow is.\

TBD sensor specs.\

TBD måske lille skema over specs?

\section{Analog-to-digital converter}


\subsubsection{ADC diagram} 

\subsubsection{The usage of ADC}

\section{The chipKIT Uno32 board}


\section{The motor shield - PKA03}

\subsection{The H bridge}
The robot will make use of an H-bridge. An H-bridge is a circuit made for controlling the motor of the robot, by making sure the motor will never try to do forward and backward motion  and cause errors. The point of using an H-bridge is to ensure motor safety and functionality.

\section{The Bluetooth tranceiver}

