
Beskriv Software section

\subsection{Software diagram}

\section{Analog to digital conversion}
%\begin{lstlisting}
%\end{lstlisting}

\section{Pulse-width modulation}

\subsection {Why utilize pulse-width modulation}

Pulse-width modulation, or PWM, is a way to regulate power distribution within a system. It is a software solution that manages when a device receives power, and for how long at a time it does this. This is called a duty cycle. The robot utilizes PWM for its motors, to regulate how quickly it moves. PWM can be compared to turning a switch on and off extremely quickly - much more quickly than what will affect the performance of the motors. Effectively, this means that the robot's programming will now be able to regulate speed autonomously. 
 
\subsection {Duty cycles}

The duty cycle is used to describe how long the power is 'on' compared to 'off'. A higher duty cycle will yield more energy than a low one. The programming uses a frequency of 1000Hz, which makes it straightforward to calculate to real time, if this is needed - it also provides enough precision to make the motors responsive quickly.

\section{Part conclusion}
Software problemer og løsninger